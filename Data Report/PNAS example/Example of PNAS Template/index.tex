\documentclass[9pt,twocolumn,twoside,]{pnas-new}

% Use the lineno option to display guide line numbers if required.
% Note that the use of elements such as single-column equations
% may affect the guide line number alignment.


\usepackage[T1]{fontenc}
\usepackage[utf8]{inputenc}

% tightlist command for lists without linebreak
\providecommand{\tightlist}{%
  \setlength{\itemsep}{0pt}\setlength{\parskip}{0pt}}


% Pandoc citation processing
\newlength{\cslhangindent}
\setlength{\cslhangindent}{1.5em}
\newlength{\csllabelwidth}
\setlength{\csllabelwidth}{3em}
\newlength{\cslentryspacingunit} % times entry-spacing
\setlength{\cslentryspacingunit}{\parskip}
% for Pandoc 2.8 to 2.10.1
\newenvironment{cslreferences}%
  {}%
  {\par}
% For Pandoc 2.11+
\newenvironment{CSLReferences}[2] % #1 hanging-ident, #2 entry spacing
 {% don't indent paragraphs
  \setlength{\parindent}{0pt}
  % turn on hanging indent if param 1 is 1
  \ifodd #1
  \let\oldpar\par
  \def\par{\hangindent=\cslhangindent\oldpar}
  \fi
  % set entry spacing
  \setlength{\parskip}{#2\cslentryspacingunit}
 }%
 {}
\usepackage{calc}
\newcommand{\CSLBlock}[1]{#1\hfill\break}
\newcommand{\CSLLeftMargin}[1]{\parbox[t]{\csllabelwidth}{#1}}
\newcommand{\CSLRightInline}[1]{\parbox[t]{\linewidth - \csllabelwidth}{#1}\break}
\newcommand{\CSLIndent}[1]{\hspace{\cslhangindent}#1}


\templatetype{pnasresearcharticle}  % Choose template

\title{Observations in ice-rich permafrost systems, Prudhoe Bay Alaska,
2020-2021}

\author[1]{Donald A. Walker}
\author[2]{Amy L. Breen}
\author[3]{Anja N. Kade}
\author[4]{Mikhail Kanevskiy}
\author[5]{Dmitry J. Nicolsky}
\author[6]{Ronald P. Daanen}
\author[4]{Benjamin M. Jones}
\author[7]{Helena Bergstedt}
\author[1]{Emily Watson-Cook}
\author[8]{Jana L. Peirce}

  \affil[1]{University of Alaska Fairbanks, Institute of Arctic Biology,
Alaska Geobotany Center, and Department of Biology and Wildlife,
Fairbanks, Alaska}
  \affil[2]{University of Alaska Fairbanks, International Arctic
Research Center, Fairbanks, Alaska}
  \affil[3]{University of Alaska Fairbanks, Department of Biology and
Wildlife, Fairbanks, Alaska}
  \affil[4]{University of Alaska Fairbanks, Institute of Northern
Engineering, Fairbanks, Alaska}
  \affil[5]{University of Alaska Fairbanks, Geophysical Institute,
Fairbanks, Alaska}
  \affil[6]{Alaska Division of Geological \& Geophysical Surveys,
Department of Natural Resources, Fairbanks, Alaska}
  \affil[7]{b.geos GmbH, Vienna, Austria}
  \affil[8]{University of Alaska Fairbanks}


% Please give the surname of the lead author for the running footer
\leadauthor{Walker}

% Please add here a significance statement to explain the relevance of your work
\significancestatement{}


\authorcontributions{}



\correspondingauthor{\textsuperscript{1} To whom correspondence should
be addressed. Email:
\href{mailto:dawalker@alaska.edu}{\nolinkurl{dawalker@alaska.edu}}}

% Keywords are not mandatory, but authors are strongly encouraged to provide them. If provided, please include two to five keywords, separated by the pipe symbol, e.g:
 \keywords{  ice-rich permafrost |  arctic vegetation |  prudhoe bay,
alaska  } 

\begin{abstract}
The National Science Foundation's Navigating the New Arctic (NNA)
project'' ``Landscape evolution and adapting to change in Ice-Rich
Permafrost Systems (NNA-IRPS)'' is focused on ice-rich permafrost
systems. This data report covers field seasons in 2020 and 2021 at the
NNA-IRPS field sites in the Prudhoe Bay Oilfield (PBO). The 2020 field
season was abbreviated because of the Covid restrictions on travel and
access to hotel facilities in the PBO.The primary goals were to (1)
conduct a reconnaissance of a new Natural Ice-Rich Permafrost
Observatory (NIRPO), (2) monitor late-season thaw depths, water-depths,
ice-wedge polygon microrelief contrasts, and vegetation distribution
along six previously established transects in the PBO, and (3) provide
training and field-site overview for a new graduate student and post
doc.The 2021 field season focused on baseline information for the NIRPO
site. An overview of the tasks, field team, schedule, and logistics is
followed by sections devoted to summaries of (1) remote sensing
activities (Daanen and Jones), (2) observations along transects at the
NIRPO and other PBO transects (Walker et al.), (3) observations from the
NIRPO terrestrial plots (Walker and Breen), (4) thermokarst-pond
vegetation and environments (Watson-Cook), (5) trace-gas fluxes (Kade),
(6) basal-peat dating (Bergstedt), (7) permafrost borehole temperature
stations (Nicolsky and Romanovsky), and (8) studies of permafrost
cryostructure (Kanevskiy and Shur). Results of some preliminary analyses
are presented with these summaries. Tables containing several of the
datasets are in the appendices with instructions on where to access the
data in the Arctic Data Center.
\end{abstract}

\dates{This manuscript was compiled on \today}
\doi{\url{www.pnas.org/cgi/doi/10.1073/pnas.XXXXXXXXXX}}

\begin{document}

% Optional adjustment to line up main text (after abstract) of first page with line numbers, when using both lineno and twocolumn options.
% You should only change this length when you've finalised the article contents.
\verticaladjustment{-2pt}



\maketitle
\thispagestyle{firststyle}
\ifthenelse{\boolean{shortarticle}}{\ifthenelse{\boolean{singlecolumn}}{\abscontentformatted}{\abscontent}}{}

% If your first paragraph (i.e. with the \dropcap) contains a list environment (quote, quotation, theorem, definition, enumerate, itemize...), the line after the list may have some extra indentation. If this is the case, add \parshape=0 to the end of the list environment.

\acknow{This project was primarily funded by the National Science
Foundation NNA Award (1928237) and built on a previous award from the
NSF ArcSEES program (1263854) with contributions the Bureau of Ocean
Energy Management, Bureau of Land Management, and U.S. Geological
Survey. The aerial surveys were made with the collaboration of the
Alaska Division of Geological and Geophysical Surveys. The participation
of several of members of the expedition was possible because of other
NSF awards to Ben Jones and colleagues (1806213), Yuri Shur/Mikhail
Kanevskiy (820883), and Vladimir Romanovsky/Dmitry Nicolsky (1832238,
1927708). Logistics support was provided the Battelle Arctic Research
Operations office in Fairbanks, and the Institute of Arctic Biology,
University of Alaska Fairbanks.}

This PNAS journal template is provided to help you write your work in
the correct journal format. Instructions for use are provided below.

Note: please start your introduction without including the word
``Introduction'' as a section heading (except for math articles in the
Physical Sciences section); this heading is implied in the first
paragraphs.

\hypertarget{introduction}{%
\section{Introduction}\label{introduction}}

\hypertarget{description-of-the-study-area}{%
\subsection{Description of the study
area}\label{description-of-the-study-area}}

The 2020 and 2021 field seasons were focused in the Prudhoe Bay Area in
the vicinity of Lake Colleen and the Deadhorse Airport at three
already-established IRPS study sites; the Jorgenson Site (JS), Colleen
Site (CS), Airport site (AS), and a new Natural Ice-Rich Permafrost
Observatory (NIRPO).

\hypertarget{author-affiliations}{%
\section{Author Affiliations}\label{author-affiliations}}

Include department, institution, and complete address, with the
ZIP/postal code, for each author. Use lower case letters to match
authors with institutions, as shown in the example. Authors with an
ORCID ID may supply this information at submission.

\hypertarget{submitting-manuscripts}{%
\subsection{Submitting Manuscripts}\label{submitting-manuscripts}}

All authors must submit their articles at
\href{http://www.pnascentral.org/cgi-bin/main.plex}{PNAScentral}. If you
are using Overleaf to write your article, you can use the ``Submit to
PNAS'' option in the top bar of the editor window.

\hypertarget{format}{%
\subsection*{Format}\label{format}}
\addcontentsline{toc}{subsection}{Format}

Many authors find it useful to organize their manuscripts with the
following order of sections; Title, Author Affiliation, Keywords,
Abstract, Significance Statement, Results, Discussion, Materials and
methods, Acknowledgments, and References. Other orders and headings are
permitted.

\hypertarget{manuscript-length}{%
\subsection{Manuscript Length}\label{manuscript-length}}

PNAS generally uses a two-column format averaging 67 characters,
including spaces, per line. The maximum length of a Direct Submission
research article is six pages and a PNAS PLUS research article is ten
pages including all text, spaces, and the number of characters displaced
by figures, tables, and equations. When submitting tables, figures,
and/or equations in addition to text, keep the text for your manuscript
under 39,000 characters (including spaces) for Direct Submissions and
72,000 characters (including spaces) for PNAS PLUS.

\hypertarget{references}{%
\subsection{References}\label{references}}

References should be cited in numerical order as they appear in text;
this will be done automatically via bibtex, e.g. (1) and (2, 3). All
references, including for the SI, should be included in the main
manuscript file. References appearing in both sections should not be
duplicated. SI references included in tables should be included with the
main reference section.

\hypertarget{data-archival}{%
\subsection{Data Archival}\label{data-archival}}

PNAS must be able to archive the data essential to a published article.
Where such archiving is not possible, deposition of data in public
databases, such as GenBank, ArrayExpress, Protein Data Bank, Unidata,
and others outlined in the Information for Authors, is acceptable.

\hypertarget{language-editing-services}{%
\subsection{Language-Editing Services}\label{language-editing-services}}

Prior to submission, authors who believe their manuscripts would benefit
from professional editing are encouraged to use a language-editing
service (see list at www.pnas.org/site/authors/language-editing.xhtml).
PNAS does not take responsibility for or endorse these services, and
their use has no bearing on acceptance of a manuscript for publication.

\begin{figure}
\centering
\includegraphics{frog.png}
\caption{Placeholder image of a frog with a long example caption to show
justification setting.{}}
\end{figure}

\hypertarget{sec:figures}{%
\subsection{Digital Figures}\label{sec:figures}}

Only TIFF, EPS, and high-resolution PDF for Mac or PC are allowed for
figures that will appear in the main text, and images must be final
size. Authors may submit U3D or PRC files for 3D images; these must be
accompanied by 2D representations in TIFF, EPS, or high-resolution PDF
format. Color images must be in RGB (red, green, blue) mode. Include the
font files for any text.

Figures and Tables should be labelled and referenced in the standard way
using the \texttt{\textbackslash{}label\{\}} and
\texttt{\textbackslash{}ref\{\}} commands.

Figure \[fig:frog\] shows an example of how to insert a column-wide
figure. To insert a figure wider than one column, please use the
\texttt{\textbackslash{}begin\{figure*\}...\textbackslash{}end\{figure*\}}
environment. Figures wider than one column should be sized to 11.4 cm or
17.8 cm wide.

\hypertarget{single-column-equations}{%
\subsection*{Single column equations}\label{single-column-equations}}
\addcontentsline{toc}{subsection}{Single column equations}

Authors may use 1- or 2-column equations in their article, according to
their preference.

To allow an equation to span both columns, options are to use the
\texttt{\textbackslash{}begin\{figure*\}...\textbackslash{}end\{figure*\}}
environment mentioned above for figures, or to use the
\texttt{\textbackslash{}begin\{widetext\}...\textbackslash{}end\{widetext\}}
environment as shown in equation \[eqn:example\] below.

Please note that this option may run into problems with floats and
footnotes, as mentioned in the \href{http://texdoc.net/pkg/cuted}{cuted
package documentation}. In the case of problems with footnotes, it may
be possible to correct the situation using commands
\texttt{\textbackslash{}footnotemark} and
\texttt{\textbackslash{}footnotetext}.

\[\begin{aligned}
(x+y)^3&=(x+y)(x+y)^2\\
       &=(x+y)(x^2+2xy+y^2) \label{eqn:example} \\
       &=x^3+3x^2y+3xy^3+x^3. 
\end{aligned}\]

\hypertarget{supporting-information-si}{%
\subsection{Supporting Information
(SI)}\label{supporting-information-si}}

The main text of the paper must stand on its own without the SI. Refer
to SI in the manuscript at an appropriate point in the text. Number
supporting figures and tables starting with S1, S2, etc. Authors are
limited to no more than 10 SI files, not including movie files. Authors
who place detailed materials and methods in SI must provide sufficient
detail in the main text methods to enable a reader to follow the logic
of the procedures and results and also must reference the online
methods. If a paper is fundamentally a study of a new method or
technique, then the methods must be described completely in the main
text. Because PNAS edits SI and composes it into a single PDF, authors
must provide the following file formats only.

\hypertarget{si-text}{%
\subsubsection{SI Text}\label{si-text}}

Supply Word, RTF, or LaTeX files (LaTeX files must be accompanied by a
PDF with the same file name for visual reference).

\hypertarget{si-figures}{%
\subsubsection{SI Figures}\label{si-figures}}

Provide a brief legend for each supporting figure after the supporting
text. Provide figure images in TIFF, EPS, high-resolution PDF, JPEG, or
GIF format; figures may not be embedded in manuscript text. When saving
TIFF files, use only LZW compression; do not use JPEG compression. Do
not save figure numbers, legends, or author names as part of the image.
Composite figures must be pre-assembled.

\hypertarget{d-figures}{%
\subsubsection{3D Figures}\label{d-figures}}

Supply a composable U3D or PRC file so that it may be edited and
composed. Authors may submit a PDF file but please note it will be
published in raw format and will not be edited or composed.

\hypertarget{si-tables}{%
\subsubsection{SI Tables}\label{si-tables}}

Supply Word, RTF, or LaTeX files (LaTeX files must be accompanied by a
PDF with the same file name for visual reference); include only one
table per file. Do not use tabs or spaces to separate columns in Word
tables.

\hypertarget{si-datasets}{%
\subsubsection{SI Datasets}\label{si-datasets}}

Supply Excel (.xls), RTF, or PDF files. This file type will be published
in raw format and will not be edited or composed.

\hypertarget{si-movies}{%
\subsubsection{SI Movies}\label{si-movies}}

Supply Audio Video Interleave (avi), Quicktime (mov), Windows Media
(wmv), animated GIF (gif), or MPEG files and submit a brief legend for
each movie in a Word or RTF file. All movies should be submitted at the
desired reproduction size and length. Movies should be no more than 10
MB in size.

\hypertarget{still-images}{%
\subsubsection{Still images}\label{still-images}}

Authors must provide a still image from each video file. Supply TIFF,
EPS, high-resolution PDF, JPEG, or GIF files.

\hypertarget{appendices}{%
\subsubsection{Appendices}\label{appendices}}

PNAS prefers that authors submit individual source files to ensure
readability. If this is not possible, supply a single PDF file that
contains all of the SI associated with the paper. This file type will be
published in raw format and will not be edited or composed.

\showmatmethods
\showacknow
\pnasbreak

\hypertarget{refs}{}
\begin{CSLReferences}{0}{0}
\leavevmode\vadjust pre{\hypertarget{ref-belkin2002using}{}}%
\CSLLeftMargin{1. }
\CSLRightInline{Belkin M, Niyogi P (2002) Using manifold stucture for
partially labeled classification. \emph{Advances in Neural Information
Processing Systems}, pp 929--936.}

\leavevmode\vadjust pre{\hypertarget{ref-berard1994embedding}{}}%
\CSLLeftMargin{2. }
\CSLRightInline{Bérard P, Besson G, Gallot S (1994) Embedding riemannian
manifolds by their heat kernel. \emph{Geometric \& Functional Analysis
GAFA} 4(4):373--398.}

\leavevmode\vadjust pre{\hypertarget{ref-coifman2005geometric}{}}%
\CSLLeftMargin{3. }
\CSLRightInline{Coifman RR, et al. (2005) Geometric diffusions as a tool
for harmonic analysis and structure definition of data: Diffusion maps.
\emph{Proceedings of the National Academy of Sciences of the United
States of America} 102(21):7426--7431.}

\end{CSLReferences}



% Bibliography
% \bibliography{pnas-sample}

\end{document}
